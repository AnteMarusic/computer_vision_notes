\section{Basic definitions}
Formally: extract semantic
and geometric information
from an image.


\begin{enumerate}
    \item Semantic information
          \begin{enumerate}
              \item What objects are in the image?
              \item Who is the person?
              \item What is the person doing?
          \end{enumerate}
    \item Geometric information
          \begin{enumerate}
              \item Where are objects in the image?
              \item How far is the person from me?
              \item What is the shape of the car?
              \item In what direction is the car moving?
              \item Eyes of AI
          \end{enumerate}
\end{enumerate}


\paragraph*{goal of computer vision}
\begin{enumerate}
    \item Compute the 3D shape of the world
    \item Recognize objects and people
    \item Improve photos (“Computational Photography”)
\end{enumerate}


\paragraph*{Challenges of computer vision}
\begin{enumerate}
    \item Viewpoint variation
    \item Illumination
    \item Scale
    \item Intra-class variation (elements of the same class can vary a lot (example in the class of ferraris there are newer and older models))
    \item Motion (The subject in the image might be moving fast)
    \item Occlusion (The subject in the immage might be partially hidden behind other objects)
    \item Background clutter (Difficult to distinguish the subject form a cluttery background)
\end{enumerate}

Computer vision uses knowledge from
\begin{enumerate}
    \item physics
    \item math
    \item Statistical learning
\end{enumerate}


\paragraph*{Related fields are:}
\begin{enumerate}
    \item image processing
    \item computer graphics
\end{enumerate}


\paragraph*{What is an image}
\begin{itemize}
    \item A grid/matrix of intensity values.
    \item We can think of a (grayscale, continuous)
          image as a function, f, from R2 to R: f (x,y) gives
          the intensity at position (x,y)
    \item A digital image is a discrete (sampled, quantized)
          version of this function
\end{itemize}


\begin{itemize}
    \item quantization is lossy
    \item As with any function, we can apply operators to an image
\end{itemize}


\paragraph*{Current standard}
\begin{enumerate}
    \item grayscale image: 1 matrix
    \item RGB image: 3 matrices
    \item (common to use one byte per value: 0 = black, 255 = white)
          1 byte per color means 24 bits that are 16ml colors
\end{enumerate}

Since 2d images are projections of 3d images there is a loss of information.
To rebuild the 3d scenario we need additional prior information on the context or
structure of the world.



\subsection[short]{Filters}
filters are transformations of the image (since the image is a function we can
apply transformations to it).\\
Filters are a key component of convolutional NN.

\paragraph*{Linear filters}
Replace each pixel by a linear combination (a weighted
sum) of its neighbors.
The weights for the linear combination is called the
“kernel” (or “mask”, “filter”).

Includes cross-correlation and convolution.

\paragraph*{Cross-correlation}
Let be the image, be the kernel (of
size 2k+1 x 2k+1), and be the output
image

\[
    G[i,j] = \sum_{u=-k}^{k} \sum_{v=-k}^{k} H[u,v] F[i+u, j+v]
\]

In short form:
\[
    G = H \otimes F
\]

\paragraph*{Convolution}
Same as cross-correlation, except that the
kernel is “flipped” (horizontally and vertically)

\[
    G[i,j] = \sum_{u=-k}^{k} \sum_{v=-k}^{k} H[u,v] F[i-u, j-v]
\]

In short form:
\[
    G = H * F
\]

\paragraph*{Properties}
\begin{itemize}
    \item Convolution is commutative and associative.
    \item Correlation is not associative
\end{itemize}


\paragraph*{Complexity of convolution}
Suppose the resolution of a grayscale image is HxW, the kernel size is KxK, and
padding is used to retain the image resolution.
\[HWK^2\]


\paragraph*{Examples}
\paragraph*{Mean filtering}

\[
    \frac{1}{9}
    \begin{bmatrix}
        1 & 1 & 1 \\
        1 & 1 & 1 \\
        1 & 1 & 1
    \end{bmatrix}
\]
Goal: Removes “high-frequency” components from the
image (low-pass filter) (Images become more smooth)


\paragraph*{Identical image}
\[
    \frac{1}{9}
    \begin{bmatrix}
        0 & 0 & 0 \\
        0 & 1 & 0 \\
        0 & 0 & 0
    \end{bmatrix}
\]


\paragraph*{Image shifted left by one pixel}
\[
    \frac{1}{9}
    \begin{bmatrix}
        0 & 0 & 0 \\
        1 & 0 & 0 \\
        0 & 0 & 0
    \end{bmatrix}
\]


\paragraph*{sharpening filter}
\[
    \left(
    \begin{bmatrix}
        0 & 0 & 0 \\
        0 & 2 & 0 \\
        0 & 0 & 0
    \end{bmatrix}
    -
    \frac{1}{9}
    \begin{bmatrix}
        1 & 1 & 1 \\
        1 & 1 & 1 \\
        1 & 1 & 1
    \end{bmatrix}
    \right)
    =
\]


\paragraph*{Detail extraction and sharpening revisited}
\begin{figure}[H]
    \centering
    \includegraphics*[width=0.5\textwidth]{Pictures/High_passing.png}
    \caption{Sharpening}
    \label{<label>}
\end{figure}


\paragraph*{Gaussian kernel}
\[
    G_\sigma = \frac{1}{2\pi\sigma^2} e^{-\frac{x^2 + y^2}{2\sigma^2}}
\]
Multiply the kernel for constant factor at front makes volume sum to 1\\

Goal: Removes “high-frequency” components from the
image (low-pass filter) (Images become more smooth)\\

Convolution with self is another Gaussian.


\paragraph*{Non linear filters}
\paragraph*{Median filter}
Replace each pixel with the median value of the window.

\begin{itemize}
    \item Is a median filter a kind of convolution?
          No is a non linear filter that applies the median operation.
          It can't be recreated by multipling a matrix
    \item More robust average than the mean
    \item a single very noisy pixel in a neighborhood will not
          affect the median value significantly.
    \item Better at preserving sharp edges than the
          mean filter
    \item does not create new unrealistic pixel values when
          the filter straddles an edge
\end{itemize}

Usage:
\begin{itemize}
    \item Used for noise reduction
          because it effectively removes outliers while preserving edges.
    \item Advantage for removing salt and pepper pixels
    \item The best thing would be to take lots of image and average them
\end{itemize}

\paragraph*{Comparison between convolution and Median filter}
\begin{itemize}
    \item Convolution is used for
          Blurring, sharpening, edge detection.
    \item Median is for
          Noise reduction (especially salt-and-pepper noise).
\end{itemize}

